\documentclass[12pt,a4paper]{article}

\usepackage[utf8]{inputenc}
\usepackage[french]{babel}
\usepackage[T1]{fontenc}
\usepackage{amsmath}
\usepackage{amsfonts}
\usepackage{amssymb}
\usepackage[left=2cm,right=2cm,top=2cm,bottom=2cm]{geometry}
\usepackage{my_listings}

\author{Théophile \textsc{Bastian} et Noémie \textsc{Cartier}}
\title{Rapport de projet d'algorithmique -- Algorithme de \textsc{Dijkstra}}



\begin{document}
\maketitle

\section{Utilisation}

\subsection{Compilation}

Pour compiler, il faut se placer dans le dossier racine et exécuter la commande \texttt{make}~; pour réinitialiser le dossier, et supprimer les fichiers créés par \texttt{make}, on peut exécuter la commande \texttt{make clean}.

L'exécutable produit s'appelle \texttt{dijkstra}.

\subsection{Formatage de l'entrée et options}

L'entrée doit se présenter sous la forme d'un graphe orienté pondéré au format canonique tel que précisé sur la page web du projet, suivi du nombre de sources à partir desquelles on veut exécuter l'algorithme de Dijkstra, puis, pour chacune de ces sources, de son identifiant dans le graphe (son numéro de nœud).

\paragraph{Exemple}

\begin{lstlisting}
$./dijkstra
 6 10                                                                            
 0 1 5                                                                           
 1 3 0                                                                           
 1 4 1                                                                           
 1 2 3                                                                           
 2 3 5                                                                           
 2 3 3                                                                           
 3 4 0                                                                           
 3 4 1                                                                           
 4 3 4                                                                           
 4 0 5                                                                           
 1 0  
\end{lstlisting}

Les options qui peuvent être passées à \texttt{./dijkstra} sont~:
\begin{itemize}
\item \lstbash{--naive}~: applique l'algorithme de Dijkstra avec une structure de file de priorité naïve et non optimisée~;
\item \lstbash{--dists-only}~: affiche uniquement les distances entre la source et les sommets, sans indiquer le chemin à parcourir.
\end{itemize}

\subsection{Formatage de sortie}

Les sorties se présentent sous la forme d'une suite de lignes, sous uns des deux formes suivantes~:
\begin{lstlisting}
[numéro du sommet]: [distance au sommet], [chemin qui y mène]
[numéro du sommet]: +∞
\end{lstlisting}

\subsection{Outils}

\end{document}