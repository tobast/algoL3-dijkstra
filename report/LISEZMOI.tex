\documentclass[12p0t,a4paper]{article}

\usepackage[utf8]{inputenc}
\usepackage[french]{babel}
\usepackage[T1]{fontenc}
\usepackage{amsmath}
\usepackage{amsfonts}
\usepackage{amssymb}
\usepackage[left=2cm,right=2cm,top=2cm,bottom=2cm]{geometry}
\usepackage{my_listings}
\usepackage{siunitx}
\usepackage{tikz} % To generate the plot from csv
\usepackage{pgfplots}

\pgfplotsset{compat=newest} % Allows to place the legend below plot
\usepgfplotslibrary{units} % Allows to enter the units nicely

\sisetup{
  round-mode          = places,
  round-precision     = 2,
}

\author{Théophile \textsc{Bastian} et Noémie \textsc{Cartier}}
\title{Rapport de projet d'algorithmique -- Algorithme de \textsc{Dijkstra}}



\begin{document}
\maketitle

\section{Utilisation}

\subsection{Compilation}

Pour compiler, il faut se placer dans le dossier racine et exécuter la commande \lstbash{make}~; pour réinitialiser le dossier, et supprimer les fichiers créés par \lstbash{make}, on peut exécuter la commande \lstbash{make clean}.

L'exécutable produit s'appelle \lstbash{dijkstra}.

\subsection{Formatage de l'entrée et options}

L'entrée doit se présenter sous la forme d'un graphe orienté pondéré au format canonique tel que précisé sur la page web du projet, suivi du nombre de sources à partir desquelles on veut exécuter l'algorithme de Dijkstra, puis, pour chacune de ces sources, de son identifiant dans le graphe (son numéro de nœud).

\paragraph{Exemple}

\begin{lstlisting}
$./dijkstra
 6 10                                                                            
 0 1 5                                                                           
 1 3 0                                                                           
 1 4 1                                                                           
 1 2 3                                                                           
 2 3 5                                                                           
 2 3 3                                                                           
 3 4 0                                                                           
 3 4 1                                                                           
 4 3 4                                                                           
 4 0 5                                                                           
 1 0  
\end{lstlisting}

Les options qui peuvent être passées à \lstbash{./dijkstra} sont~:
\begin{itemize}
\item \lstbash{--naive}~: applique l'algorithme de Dijkstra avec une structure de file de priorité naïve et non optimisée~;
\item \lstbash{--dists-only}~: affiche uniquement les distances entre la source et les sommets, sans indiquer le chemin à parcourir~;
\item \lstbash{--print-bound}~: affiche \lstbash{bound: <n>}, où n est le nombre d'opérations effectuées supposées.
\end{itemize}

\subsection{Formatage de sortie}

Les sorties se présentent sous la forme d'une suite de lignes, sous uns des deux formes suivantes~:
\begin{lstlisting}
[numéro du sommet]: [distance au sommet], [chemin qui y mène]
[numéro du sommet]: +∞
\end{lstlisting}

\subsection{Outils}

Les outils fournis avec l'algorithme, ne faisant à strictement parler partie du projet, sont les suivants~:

\begin{itemize}
\item \lstbash{checker/checker} prend exactement la même entrée et affiche exactement la même sortie que \lstbash{./dijkstra --dists-only} (dans le cas où celui-ci fonctionne correctement). Il s'agit d'une implémentation C++ utilisant les files de priorité \lstbash{stl}, supposée correcte (ou du moins ne présentant pas les mêmes erreurs...)~;
\item \lstbash{./genRandGraph.sh <prefix>} génère un graphe aléatoire à l'aide de \lstbash{genRandGraph.py} (ce dernier contenant des constantes modifiables au début). Le graphe est présenté sous la forme de deux fichiers~: \lstbash{<prefix>.png} est une représentation graphique du graphe, tandis que \lstbash{<prefix>.in} est un fichier d'entrée de \lstbash{dijkstra} ayant pour unique source \lstbash{0}.

\textbf{Attention~!} Pour générer des graphes trop gros (plus d'une centaine de nœuds), ou simplement pour effectuer beaucoup de tests, il peut être utile de passer \lstbash{--no-dot} comme \emph{première} option à la commande afin d'éviter de générer l'image, ce qui ralentit considérablement l'exécution~;
\item \lstbash{./randChecking.sh <n>} exécute n tests aléatoires successivement, durant lesquels il compare les résultats de \lstbash{./dijkstra}, de \lstbash{./dijkstra --naive} et de \lstbash{checker/checker}. Le programme stocke les incohérences dans le dossier \lstbash{tests}.
\end{itemize}

\section{Détails d'implémentation}

\subsection{Fichiers source}

Les différents fichiers sources ont les rôles suivants~:

\begin{itemize}
\item \lstbash{genericStruct.c,h} définit un certain nombre de structures, macro\ldots{} utilisées par la suite~;

\item \lstbash{extendLists.c,h} définit une structure de tableaux d'entiers redimensionnables, proche des \lstbash{std::vector} en C++~;

\item \lstbash{tree.c,h} définit une structure d'arbre utilisée dans les tas de Fibonacci, basée sur des pointeurs \lstbash{child} et \lstbash{sibling}~;

\item \lstbash{linkedList.c,h} définit des listes doublement chaînées circulaires de pointeurs sur des \lstbash{Tree}~;

\item \lstbash{fiboHeap.c,h} définit la structure des tas de Fibonacci à partir des structures précédentes~;

\item \lstbash{graph.c,h} définit la structure des graphes à partir de listes d'adjacence basées sur \lstbash{ExtendList}~;

\item \lstbash{naiveDijkstra.c,h} implémente une version naive de l'algorithme de Dikstra à partir de files de priorité telles que décrites dans le sujet, basées sur la structure \lstbash{ExtendList}~;

\item \lstbash{dijkstra.c,h} implémente une version plus efficace de l'algorithme de Dijkstra à partir de files de priorité basées sur la structure des tas de Fibonacci.
\end{itemize}

\subsection{Tests}

Pour confirmer la correction de nos algorithmes, nous avons passé la commande

\begin{lstlisting}[language=bash]
./randChecking.sh 10000
\end{lstlisting}

sur des graphes de nombre de sommets 1000, de degré sortant des sommets compris entre 0 et 20 et d'arêtes de poids inférieur à 100 (et parfois nul).

Comme nos trois implémentations, qui sont a priori suffisamment différentes pour ne pas faire les mêmes erreurs, donnent des résultats cohérents, nous estimons que nos implémentations sont correctes.

De plus, nous avons utilisé \lstbash{valgrind} pour vérifier l'absence de fuites de mémoire dans notre programme, et ce avec succès.

\section{Performances}


\begin{figure}[h!]
  \begin{center}
    \begin{tikzpicture}
      \begin{axis}[
          width=\linewidth, % Scale the plot to \linewidth
          grid=major, % Display a grid
          grid style={dashed,gray!30}, % Set the style
          xlabel=Nombre de sommets, % Set the labels
          ylabel=Temps d'exécution,
          x unit=, % Set the respective units
          y unit=\si{\second},
          legend style={at={(0.5,-0.2)},anchor=north}, % Put the legend below the plot
          x tick label style={rotate=90,anchor=east} % Display labels sideways
        ]
        \addplot table[x=column 1,y=column 2,col sep=comma] {table.csv};
        \addplot  
        \legend{Plot}
      \end{axis}
    \end{tikzpicture}
    \caption{My first autogenerated plot.}
  \end{center}
\end{figure}

\end{document}